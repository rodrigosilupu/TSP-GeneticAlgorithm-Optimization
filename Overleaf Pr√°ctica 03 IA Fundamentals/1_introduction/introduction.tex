\section{Introducción}

En el ámbito de la optimización y búsqueda de soluciones, los algoritmos genéticos han demostrado ser herramientas efectivas debido a su capacidad para explorar vastos espacios de solución y encontrar óptimos globales. Estos algoritmos se inspiran en los principios de la evolución natural y utilizan mecanismos como la selección, cruce y mutación para generar nuevas soluciones a partir de las existentes.

\vspace{1em} 
El presente trabajo se enfoca en la implementación y comparación de diferentes métodos de selección y inicialización de población en algoritmos genéticos aplicados a un problema de optimización de rutas en un conjunto de 100 ciudades. En particular, se llevarán a cabo dos experimentos principales:

\begin{itemize}
\item Experimento 1: Implementación y comparación de la solución obtenida usando los métodos de selección: {\it Roulette wheel selection}, {\it Rank-based selection}, {\it Fitness scaling} y {\it Tournament selection}.
\item Experimento 2: Implementación y comparación de la solución obtenida usando los métodos de inicialización de población: {\it random}, {\it heuristic} y {\it hybrid initialization}.
\end{itemize}

Para cada experimento, se guardarán los valores de aptitud en intervalos específicos de generaciones y se generarán gráficos que mostrarán cómo cada método afecta la búsqueda de soluciones. 

